\hypertarget{einfuxfchrung-und-ziele}{%
\section{Einführung und Ziele}\label{einfuxfchrung-und-ziele}}

Beschreibt die wesentlichen Anforderungen und treibenden Kräfte, die bei
der Umsetzung der Softwarearchitektur und Entwicklung des Systems
berücksichtigt werden müssen.

Dazu gehören:

\begin{itemize}
\tightlist
\item
  zugrunde liegende Geschäftsziele,
\item
  wesentliche Aufgabenstellungen und
\item
  essenzielle fachliche Anforderungen an das System sowie
\item
  Qualitätsziele für die Architektur und
\item
  relevante Stakeholder und deren Erwartungshaltung.
\end{itemize}

\hypertarget{aufgabenstellung}{%
\subsection{Aufgabenstellung}\label{aufgabenstellung}}

\textbf{Inhalt}

Kurzbeschreibung der fachlichen Aufgabenstellung, treibenden Kräfte,
Extrakt (oder Abstract) der Anforderungen. Verweis auf (hoffentlich
vorliegende) Anforderungsdokumente (mit Versionsbezeichnungen und
Ablageorten).

\textbf{Motivation}

Aus Sicht der späteren Nutzung ist die Unterstützung einer fachlichen
Aufgabe oder Verbesserung der Qualität der eigentliche Beweggrund, ein
neues System zu schaffen oder ein bestehendes zu modifizieren.

\textbf{Form}

Kurze textuelle Beschreibung, eventuell in tabellarischer Use-Case Form.
Sofern vorhanden, sollte die Aufgabenstellung Verweise auf die
entsprechenden Anforderungsdokumente enthalten.

Halten Sie diese Auszüge so knapp wie möglich und wägen Sie Lesbarkeit
und Redundanzfreiheit gegeneinander ab.

\hypertarget{qualituxe4tsziele}{%
\subsection{Qualitätsziele}\label{qualituxe4tsziele}}

\textbf{Inhalt}

Die Top-3 bis Top-5 der Qualitätsziele für die Architektur, deren
Erfüllung oder Einhaltung den maßgeblichen Stakeholdern besonders
wichtig sind. Gemeint sind hier wirklich Qualitätsziele, die nicht
unbedingt mit den Zielen des Projekts übereinstimmen. Beachten Sie den
Unterschied.

\textbf{Motivation}

Weil Qualitätsziele grundlegende Architekturentscheidungen oft
maßgeblich beeinflussen, sollten Sie die für Ihre Stakeholder relevanten
Qualitätsziele kennen, möglichst konkret und operationalisierbar.

\textbf{Form}

Tabellarische Darstellung der Qualitätsziele mit möglichst konkreten
Szenarien, geordnet nach Prioritäten.

\hypertarget{stakeholder}{%
\subsection{Stakeholder}\label{stakeholder}}

\textbf{Inhalt}

Expliziter Überblick über die Stakeholder des Systems -- über alle
Personen, Rollen oder Organisationen --, die

\begin{itemize}
\tightlist
\item
  die Architektur kennen sollten oder
\item
  von der Architektur überzeugt werden müssen,
\item
  mit der Architektur oder dem Code arbeiten (z.B.
  Schnittstellennutzen),
\item
  die Dokumentation der Architektur für ihre eigene Arbeit benötigen,
\item
  Entscheidungen über das System und dessen Entwicklung treffen.
\end{itemize}

\textbf{Motivation}

Sie sollten die Projektbeteiligten und -betroffenen kennen, sonst
erleben Sie später im Entwicklungsprozess Überraschungen. Diese
Stakeholder bestimmen unter anderem Umfang und Detaillierungsgrad der
von Ihnen zu leistenden Arbeit und Ergebnisse.

\textbf{Form}

Tabelle mit Rollen- oder Personennamen, sowie deren Erwartungshaltung
bezüglich der Architektur und deren Dokumentation.

\begin{longtable}[]{@{}lll@{}}
\toprule
Rolle & Kontakt & Erwartungshaltung\tabularnewline
\midrule
\endhead
\emph{\textless Rolle-1\textgreater{}} &
\emph{\textless Kontakt-1\textgreater{}} &
\emph{\textless Erwartung-1\textgreater{}}\tabularnewline
\emph{\textless Rolle-2\textgreater{}} &
\emph{\textless Kontakt-2\textgreater{}} &
\emph{\textless Erwartung-2\textgreater{}}\tabularnewline
\bottomrule
\end{longtable}
