\hypertarget{kontextabgrenzung}{%
\section{Kontextabgrenzung}\label{kontextabgrenzung}}

\textbf{Inhalt}

Die Kontextabgrenzung grenzt das System von allen
Kommunikationsbeziehungen (Nachbarsystemen und Benutzerrollen) ab. Sie
legt damit die externen Schnittstellen fest.

Differenzieren Sie fachliche (fachliche Ein- und Ausgaben) und
technische Kontexte (Kanäle, Protokolle, Hardware), falls nötig.

\textbf{Motivation}

Die fachlichen und technischen Schnittstellen zur Kommunikation gehören
zu den kritischsten Aspekten eines Systems. Stellen Sie sicher, dass Sie
diese komplett verstanden haben.

\textbf{Form}

Verschiedene Optionen:

\begin{itemize}
\item
  Diverse Kontextdiagramme
\item
  Listen von Kommunikationsbeziehungen mit deren Schnittstellen
\end{itemize}

\hypertarget{fachlicher-kontext}{%
\subsection{Fachlicher Kontext}\label{fachlicher-kontext}}

\textbf{Inhalt}

Festlegung \textbf{aller} Kommunikationsbeziehungen (Nutzer, IT-Systeme,
\ldots) mit Erklärung der fachlichen Ein- und Ausgabedaten oder
Schnittstellen. Zusätzlich (bei Bedarf) fachliche Datenformate oder
Protokolle der Kommunikation mit den Nachbarsystemen.

\textbf{Motivation}

Alle Beteiligten müssen verstehen, welche fachlichen Informationen mit
der Umwelt ausgetauscht werden.

\textbf{Form}

Alle Diagrammarten, die das System als Blackbox darstellen und die
fachlichen Schnittstellen zu den Nachbarsystemen beschreiben.

Alternativ oder ergänzend können Sie eine Tabelle verwenden. Der Titel
gibt den Namen Ihres Systems wieder; die drei Spalten sind:
Kommunikationsbeziehung, Eingabe, Ausgabe.

\textbf{\textless Diagramm und/oder Tabelle\textgreater{}}

\textbf{\textless optional: Erläuterung der externen fachlichen
Schnittstellen\textgreater{}}

\hypertarget{technischer-kontext}{%
\subsection{Technischer Kontext}\label{technischer-kontext}}

\textbf{Inhalt}

Technische Schnittstellen (Kanäle, Übertragungsmedien) zwischen dem
System und seiner Umwelt. Zusätzlich eine Erklärung (\emph{mapping}),
welche fachlichen Ein- und Ausgaben über welche technischen Kanäle
fließen.

\textbf{Motivation}

Viele Stakeholder treffen Architekturentscheidungen auf Basis der
technischen Schnittstellen des Systems zu seinem Kontext.

Insbesondere bei der Entwicklung von Infrastruktur oder Hardware sind
diese technischen Schnittstellen durchaus entscheidend.

\textbf{Form}

Beispielsweise UML Deployment-Diagramme mit den Kanälen zu
Nachbarsystemen, begleitet von einer Tabelle, die Kanäle auf
Ein-/Ausgaben abbildet.

\textbf{\textless Diagramm oder Tabelle\textgreater{}}

\textbf{\textless optional: Erläuterung der externen technischen
Schnittstellen\textgreater{}}

\textbf{\textless Mapping fachliche auf technische
Schnittstellen\textgreater{}}
