\hypertarget{laufzeitsicht}{%
\section{Laufzeitsicht}\label{laufzeitsicht}}

\textbf{Inhalt}

Diese Sicht erklärt konkrete Abläufe und Beziehungen zwischen Bausteinen
in Form von Szenarien aus den folgenden Bereichen:

\begin{itemize}
\tightlist
\item
  Wichtige Abläufe oder \emph{Features}: Wie führen die Bausteine der
  Architektur die wichtigsten Abläufe durch?
\item
  Interaktionen an kritischen externen Schnittstellen: Wie arbeiten
  Bausteine mit Nutzern und Nachbarsystemen zusammen?
\item
  Betrieb und Administration: Inbetriebnahme, Start, Stop.
\item
  Fehler- und Ausnahmeszenarien
\end{itemize}

Anmerkung: Das Kriterium für die Auswahl der möglichen Szenarien (d.h.
Abläufe) des Systems ist deren Architekturrelevanz. Es geht nicht darum,
möglichst viele Abläufe darzustellen, sondern eine angemessene Auswahl
zu dokumentieren.

\textbf{Motivation}

Sie sollten verstehen, wie (Instanzen von) Bausteine(n) Ihres Systems
ihre jeweiligen Aufgaben erfüllen und zur Laufzeit miteinander
kommunizieren.

Nutzen Sie diese Szenarien in der Dokumentation hauptsächlich für eine
verständlichere Kommunikation mit denjenigen Stakeholdern, die die
statischen Modelle (z.B. Bausteinsicht, Verteilungssicht) weniger
verständlich finden.

\textbf{Form}

Für die Beschreibung von Szenarien gibt es zahlreiche
Ausdrucksmöglichkeiten. Nutzen Sie beispielsweise:

\begin{itemize}
\tightlist
\item
  Nummerierte Schrittfolgen oder Aufzählungen in Umgangssprache
\item
  Aktivitäts- oder Flussdiagramme
\item
  Sequenzdiagramme
\item
  BPMN (Geschäftsprozessmodell und -notation) oder EPKs
  (Ereignis-Prozessketten)
\item
  Zustandsautomaten
\item
  \ldots{}
\end{itemize}

\hypertarget{bezeichnung-laufzeitszenario-1}{%
\subsection{\texorpdfstring{\emph{\textless Bezeichnung Laufzeitszenario
1\textgreater{}}}{\textless Bezeichnung Laufzeitszenario 1\textgreater{}}}\label{bezeichnung-laufzeitszenario-1}}

\begin{itemize}
\tightlist
\item
  \textless hier Laufzeitdiagramm oder Ablaufbeschreibung
  einfügen\textgreater{}
\item
  \textless hier Besonderheiten bei dem Zusammenspiel der Bausteine in
  diesem Szenario erläutern\textgreater{}
\end{itemize}

\hypertarget{bezeichnung-laufzeitszenario-2}{%
\subsection{\texorpdfstring{\emph{\textless Bezeichnung Laufzeitszenario
2\textgreater{}}}{\textless Bezeichnung Laufzeitszenario 2\textgreater{}}}\label{bezeichnung-laufzeitszenario-2}}

\ldots{}

\hypertarget{bezeichnung-laufzeitszenario-n}{%
\subsection{\texorpdfstring{\emph{\textless Bezeichnung Laufzeitszenario
n\textgreater{}}}{\textless Bezeichnung Laufzeitszenario n\textgreater{}}}\label{bezeichnung-laufzeitszenario-n}}

\ldots{}
