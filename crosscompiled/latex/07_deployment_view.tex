\hypertarget{verteilungssicht}{%
\section{Verteilungssicht}\label{verteilungssicht}}

\textbf{Inhalt}

Die Verteilungssicht beschreibt:

\begin{enumerate}
\def\labelenumi{\arabic{enumi}.}
\tightlist
\item
  die technische Infrastruktur, auf der Ihr System ausgeführt wird, mit
  Infrastrukturelementen wie Standorten, Umgebungen, Rechnern,
  Prozessoren, Kanälen und Netztopologien sowie sonstigen Bestandteilen,
  und
\item
  die Abbildung von (Software-)Bausteinen auf diese Infrastruktur.
\end{enumerate}

Häufig laufen Systeme in unterschiedlichen Umgebungen, beispielsweise
Entwicklung-/Test- oder Produktionsumgebungen. In solchen Fällen sollten
Sie alle relevanten Umgebungen aufzeigen.

Nutzen Sie die Verteilungssicht insbesondere dann, wenn Ihre Software
auf mehr als einem Rechner, Prozessor, Server oder Container abläuft
oder Sie Ihre Hardware sogar selbst konstruieren.

Aus Softwaresicht genügt es, auf die Aspekte zu achten, die für die
Softwareverteilung relevant sind. Insbesondere bei der
Hardwareentwicklung kann es notwendig sein, die Infrastruktur mit
beliebigen Details zu beschreiben.

\textbf{Motivation}

Software läuft nicht ohne Infrastruktur. Diese zugrundeliegende
Infrastruktur beeinflusst Ihr System und/oder querschnittliche
Lösungskonzepte, daher müssen Sie diese Infrastruktur kennen.

\textbf{Form}

Das oberste Verteilungsdiagramm könnte bereits in Ihrem technischen
Kontext enthalten sein, mit Ihrer Infrastruktur als EINE Blackbox. Jetzt
zoomen Sie in diese Infrastruktur mit weiteren Verteilungsdiagrammen
hinein:

\begin{itemize}
\item
  Die UML stellt mit Verteilungsdiagrammen (Deployment diagrams) eine
  Diagrammart zur Verfügung, um diese Sicht auszudrücken. Nutzen Sie
  diese, evtl. auch geschachtelt, wenn Ihre Verteilungsstruktur es
  verlangt.
\item
  Falls Ihre Infrastruktur-Stakeholder andere Diagrammarten bevorzugen,
  die beispielsweise Prozessoren und Kanäle zeigen, sind diese hier
  ebenfalls einsetzbar.
\end{itemize}

\hypertarget{infrastruktur-ebene-1}{%
\subsection{Infrastruktur Ebene 1}\label{infrastruktur-ebene-1}}

An dieser Stelle beschreiben Sie (als Kombination von Diagrammen mit
Tabellen oder Texten):

\begin{itemize}
\item
  die Verteilung des Gesamtsystems auf mehrere Standorte, Umgebungen,
  Rechner, Prozessoren o. Ä., sowie die physischen Verbindungskanäle
  zwischen diesen,
\item
  wichtige Begründungen für diese Verteilungsstruktur,
\item
  Qualitäts- und/oder Leistungsmerkmale dieser Infrastruktur,
\item
  Zuordnung von Softwareartefakten zu Bestandteilen der Infrastruktur
\end{itemize}

Für mehrere Umgebungen oder alternative Deployments kopieren Sie diesen
Teil von arc42 für alle wichtigen Umgebungen/Varianten.

\textbf{\emph{\textless Übersichtsdiagramm\textgreater{}}}

\begin{description}
\item[Begründung]
\emph{\textless Erläuternder Text\textgreater{}}
\item[Qualitäts- und/oder Leistungsmerkmale]
\emph{\textless Erläuternder Text\textgreater{}}
\item[Zuordnung von Bausteinen zu Infrastruktur]
\emph{\textless Beschreibung der Zuordnung\textgreater{}}
\end{description}

\hypertarget{infrastruktur-ebene-2}{%
\subsection{Infrastruktur Ebene 2}\label{infrastruktur-ebene-2}}

An dieser Stelle können Sie den inneren Aufbau (einiger)
Infrastrukturelemente aus Ebene 1 beschreiben.

Für jedes Infrastrukturelement kopieren Sie die Struktur aus Ebene 1.

\hypertarget{infrastrukturelement-1}{%
\subsubsection{\texorpdfstring{\emph{\textless Infrastrukturelement
1\textgreater{}}}{\textless Infrastrukturelement 1\textgreater{}}}\label{infrastrukturelement-1}}

\emph{\textless Diagramm + Erläuterungen\textgreater{}}

\hypertarget{infrastrukturelement-2}{%
\subsubsection{\texorpdfstring{\emph{\textless Infrastrukturelement
2\textgreater{}}}{\textless Infrastrukturelement 2\textgreater{}}}\label{infrastrukturelement-2}}

\emph{\textless Diagramm + Erläuterungen\textgreater{}}

\ldots{}

\hypertarget{infrastrukturelement-n}{%
\subsubsection{\texorpdfstring{\emph{\textless Infrastrukturelement
n\textgreater{}}}{\textless Infrastrukturelement n\textgreater{}}}\label{infrastrukturelement-n}}

\emph{\textless Diagramm + Erläuterungen\textgreater{}}
