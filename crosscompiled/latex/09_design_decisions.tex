\hypertarget{entwurfsentscheidungen}{%
\section{Entwurfsentscheidungen}\label{entwurfsentscheidungen}}

\textbf{Inhalt}

Wichtige, teure, große oder riskante Architektur- oder
Entwurfsentscheidungen inklusive der jeweiligen Begründungen. Mit
"Entscheidungen" meinen wir hier die Auswahl einer von mehreren
Alternativen unter vorgegebenen Kriterien.

Wägen Sie ab, inwiefern Sie Entscheidungen hier zentral beschreiben,
oder wo eine lokale Beschreibung (z.B. in der Whitebox-Sicht von
Bausteinen) sinnvoller ist. Vermeiden Sie Redundanz. Verweisen Sie evtl.
auf Abschnitt 4, wo schon grundlegende strategische Entscheidungen
beschrieben wurden.

\textbf{Motivation}

Stakeholder des Systems sollten wichtige Entscheidungen verstehen und
nachvollziehen können.

\textbf{Form}

Verschiedene Möglichkeiten:

\begin{itemize}
\tightlist
\item
  Liste oder Tabelle, nach Wichtigkeit und Tragweite der Entscheidungen
  geordnet
\item
  ausführlicher in Form einzelner Unterkapitel je Entscheidung
\item
  ADR
  (\href{http://thinkrelevance.com/blog/2011/11/15/documenting-architecture-decisions}{Architecture
  Decision Record}) für jede wichtige Entscheidung
\end{itemize}
