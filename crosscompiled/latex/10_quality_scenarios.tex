\hypertarget{qualituxe4tsanforderungen}{%
\section{Qualitätsanforderungen}\label{qualituxe4tsanforderungen}}

\textbf{Inhalt}

Dieser Abschnitt enthält möglichst alle Qualitätsanforderungen als
Qualitätsbaum mit Szenarien. Die wichtigsten davon haben Sie bereits in
Abschnitt 1.2 (Qualitätsziele) hervorgehoben.

Nehmen Sie hier auch Qualitätsanforderungen geringerer Priorität auf,
deren Nichteinhaltung oder -erreichung geringe Risiken birgt.

\textbf{Motivation}

Weil Qualitätsanforderungen die Architekturentscheidungen oft maßgeblich
beeinflussen, sollten Sie die für Ihre Stakeholder relevanten
Qualitätsanforderungen kennen, möglichst konkret und operationalisiert.

\hypertarget{qualituxe4tsbaum}{%
\subsection{Qualitätsbaum}\label{qualituxe4tsbaum}}

\textbf{Inhalt}

Der Qualitätsbaum (à la ATAM) mit Qualitätsszenarien an den Blättern.

\textbf{Motivation}

Die mit Prioritäten versehene Baumstruktur gibt Überblick über
die --- oftmals zahlreichen --- Qualitätsanforderungen.

\begin{itemize}
\item
  Baumartige Verfeinerung des Begriffes „Qualität``, mit „Qualität''
  oder „Nützlichkeit" als Wurzel.
\item
  Mindmap mit Qualitätsoberbegriffen als Hauptzweige
\end{itemize}

In jedem Fall sollten Sie hier Verweise auf die Qualitätsszenarien des
folgenden Abschnittes aufnehmen.

\hypertarget{qualituxe4tsszenarien}{%
\subsection{Qualitätsszenarien}\label{qualituxe4tsszenarien}}

\textbf{Inhalt}

Konkretisierung der (in der Praxis oftmals vagen oder impliziten)
Qualitätsanforderungen durch (Qualitäts-)Szenarien.

Diese Szenarien beschreiben, was beim Eintreffen eines Stimulus auf ein
System in bestimmten Situationen geschieht.

Wesentlich sind zwei Arten von Szenarien:

\begin{itemize}
\item
  Nutzungsszenarien (auch bekannt als Anwendungs- oder
  Anwendungsfallszenarien) beschreiben, wie das System zur Laufzeit auf
  einen bestimmten Auslöser reagieren soll. Hierunter fallen auch
  Szenarien zur Beschreibung von Effizienz oder Performance. Beispiel:
  Das System beantwortet eine Benutzeranfrage innerhalb einer Sekunde.
\item
  Änderungsszenarien beschreiben eine Modifikation des Systems oder
  seiner unmittelbaren Umgebung. Beispiel: Eine zusätzliche
  Funktionalität wird implementiert oder die Anforderung an ein
  Qualitätsmerkmal ändert sich.
\end{itemize}

\textbf{Motivation}

Szenarien operationalisieren Qualitätsanforderungen und machen deren
Erfüllung mess- oder entscheidbar.

Insbesondere wenn Sie die Qualität Ihrer Architektur mit Methoden wie
ATAM überprüfen wollen, bedürfen die in Abschnitt 1.2 genannten
Qualitätsziele einer weiteren Präzisierung bis auf die Ebene von
diskutierbaren und nachprüfbaren Szenarien.

\textbf{Form}

Entweder tabellarisch oder als Freitext.
